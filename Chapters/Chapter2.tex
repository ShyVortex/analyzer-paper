% Chapter 2

\chapter{Background and Related Work} % Main chapter title

\label{Chapter2} 

\section{Comments Quality}

Comment quality refers to how effectively developers convey information about the purpose and functionality of the code. Clear, helpful comments are critical for maintaining software over time. Without sufficient comments, source code becomes difficult to update, as developers may struggle to understand its purpose or how it works. High-quality comments should be readable, well-organized, syntactically correct, and logically clear, catering to both experienced and inexperienced developers to ensure accessibility for anyone working on the code.

\noindent Several practices can either undermine or enhance the quality of comments. For instance, when documenting long or complex methods, comments should be concise, focused, and highlight key points. Including relevant examples can also improve clarity. Additionally, proper formatting, such as consistent indentation and line breaks, is crucial for readability, as inconsistent styles or excessive spacing can cause confusion.

\noindent Problems may arise when outdated or deprecated methods are left commented out, cluttering the code and obscuring more current, useful comments. If a comment is intended to serve as documentation, it must accurately reflect the method's parameters and return types to maintain relevance and precision.

\noindent Ideally, comments should be added only when necessary. When used, they should be both concise and cohesive, avoiding redundancy or excessive detail. Inline comments that explain actions as they occur are particularly valuable, as they guide readers through the code step by step, making it easier to follow and understand.

\section{Comments Quality Analysis}
Several studies have been conducted to analyze the quality of comments across various datasets. One, presented by \textit{Khamis et al.} \cite{javadocMiner}, offers an effective and automated approach for evaluating the quality of inline documentation using a set of heuristics. Their method assesses both the linguistic quality of comments and the consistency between source code and its corresponding documentation. The researchers applied their approach to different modules of two open-source applications, \textit{ArgoUML} and \textit{Eclipse}, comparing the analysis results with bug reports from the individual modules. This comparison aimed to uncover potential connections between documentation quality and overall code quality, shedding light on how well-maintained comments may influence software reliability.

\noindent \textit{Steidl et al.} \cite{steidl2013} proposed a model for evaluating comment quality based on different comment categories. The researchers used machine learning techniques to categorize comments from \textit{Java} and \textit{C/C++} programs, allowing for a more structured analysis of comment quality. By applying metrics tailored to specific comment categories, they were able to effectively assess key quality aspects within the model, providing valuable insights into the strengths and weaknesses of code comments in these programming languages.

\noindent \textit{Tan Lin} \cite{TAN2015493} highlighted the wealth of information contained in code comments and how this can be used to enhance software maintainability and reliability. He conducted a detailed study and analysis of both free-form and semistructured comments, demonstrating their potential to improve the overall quality and robustness of software systems.

\noindent \textit{Pooja Rani} \cite{Rani2021} addressed the problem of evaluating code comments, exploring comment quality assessment from three perspectives: developer concerns, information types, and research support, identifiying a taxonomy of common concerns and challenges faced by developers when writing or generating code comments, as well as creating a table of comment quality attributes and metrics, helping to develop a unified approach for evaluating the quality of code comments based on empirical data.

\noindent \textit{He Hao} \cite{HaoHe19} investigated the patterns and practices of code commenting across different programming languages and project types. By analyzing 150 popular projects in five programming languages (\textit{JavaScript}, \textit{Java}, \textit{C++}, \textit{Python}, and \textit{Go}), the study reveals significant variations in comment density that may be influenced by the programming language and the project's purpose.

\noindent \textit{Pascarella et al.} \cite{javaCmsClassification} developed a comprehensive taxonomy for categorizing Java code comments, consisting of six top-level categories and 16 subcategories, validated through extensive manual classification of over 40,000 lines of comments from 14 Java projects, both open-source and industrial. They successfully applied supervised machine learning techniques to automatically classify code comments based on their taxonomy, achieving promising results with precision and recall rates consistently above 93\% in most categories for open-source projects, although slightly lower for industrial projects, finding that the most common comment categories were those summarizing code functionality (e.g., "SUMMARY") and providing usage instructions (e.g., "USAGE"). Comments related to development tasks or temporary changes (e.g., "TODO") were less frequent, especially in industrial projects.

\noindent \textit{Chen et al.} \cite{CHEN201945} developed a machine learning-based method using Random Forests to automatically detect the scope of comments in Java code. Their method achieved an accuracy of 81.45\% in detecting the scopes of comments across four popular open-source projects. This marked a significant improvement over heuristic-based methods, which had an accuracy of 67.10\%. The proposed method was applied to two specific tasks: outdated comment detection and automatic comment generation. In both cases, the method improved the performance of baseline approaches, demonstrating its effectiveness in enhancing software repository mining tasks.

\noindent \textit{Rani et al.}\cite{Rani2023} presented a comprehensive analysis of research on code comment quality from 2011 to 2020. The study selected and reviewed 47 relevant papers to address four key questions: (I) which types of comments researchers focus on, (II) the quality attributes (QAs) considered, (III) the tools and techniques used to evaluate comment quality, and (IV) the methods used to assess the effectiveness of these studies.