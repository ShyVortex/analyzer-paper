% Chapter 5

\chapter{Conclusion and Future Work} % Main chapter title

\label{Chapter5}

%----------------------------------------------------------------------------------------

\section{Conclusion}
We explored the critical role of code comment analysis in improving software maintainability, readability, and overall quality. Many analyses have been made to study the quality of comments, yet the majority of them focus on key specifics, one or two study scopes. We aim for more general approaches that allow to study comments quality from a variety of points of view. As such, we presented several criteria that help us divide comments in different categories, and for each category we have developed an appropriate heuristic. By leveraging natural language processing (NLP) and machine learning techniques, we evaluated the tool goodness on state-of-the-art datasets, finding average accuracy and precision scores respectively of 0.91 and 0.75 for CSN.

\section{Future Work}
This work has laid the foundation for numerous future advancements in code comment analysis. The categorizations and heuristics developed can serve as valuable resources for building machine learning models that enhance detection accuracy and extend the analysis beyond the initial scope. Future research can explore additional categories and aspects of comments that can be inferred, while also expanding the tool's applicability to a wider range of programming languages.

\noindent One of our ongoing goals is to further minimize false positives by refining the heuristics to be more precise and objective, ensuring that as few relevant instances as possible are excluded. Ultimately, this work contributes to the expanding field of automated quality assurance, helping to bridge the gap between human-readable comments and machine-driven analysis, and paving the way for more maintainable and robust codebases in the future.